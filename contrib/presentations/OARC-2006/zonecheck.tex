\include{parameters}
\usetheme{AFNIC}
\usepackage[english]{babel}
\usepackage[latin1]{inputenc}
\usepackage[T1]{fontenc}
\usepackage{bortzmeyer-utils}

\title{Zonecheck, testing a DNS zone}
\author{St�phane Bortzmeyer\\AFNIC ("\texttt{.fr}" registry)\\\texttt{bortzmeyer@nic.fr}}
\date{16 november 2006}

%\setlength{\parskip}{1ex plus 0.5ex minus 0.2ex} 
% \setlength{\parskip}{15pt} 
\setlength{\parskip}{15pt plus 10pt minus 10pt} 

\begin{document}

\maketitle

\begin{frame}
  \titlepage
\end{frame}

\begin{frame}[fragile]
 Permission is granted to copy, distribute and/or modify this document
      under the terms of the GNU Free Documentation License \url{http://www.gnu.org/licenses/licenses.html#FDL}, Version 1.2
      or any later version published by the Free Software Foundation;
      with no Invariant Sections, no Front-Cover Texts, and no Back-Cover Texts.  
\end{frame}

\begin{frame}
\frametitle{Why test a DNS zone?}

\begin{itemize}
\item<2->To make sure it works,
\item<3->To make sure it works fast (no timeouts or retransmissions).
\end{itemize}

\begin{block}<4>{It is not because ``it works'' that everything is
      perfect.}{See Ilya Sukhar's slides about the consequences of bad
      delegation.}\end{block}
\end{frame}

\begin{frame}
  \frametitle{The requirements}
We started designing the new Zonecheck in 2002 (version 2, a program by
the same name, but completely different, existed before).

The requirements for the new version were:

\begin{itemize}
\item<2->Command-line (so it can be run everywhere) and Web tool,
\item<3->Free software,
\item<4->General tool, not a small ad-hoc hack,
\item<5->Separated policy and engine (more on that later).
\end{itemize}
  
\end{frame}

\begin{frame}
\frametitle{The result}  
\begin{itemize}
\item<2->Developed by St�phane d'Alu,
\item<3->Written in Ruby,
\item<4->Available under the GPL free licence, a very important point,
  since it allows people to run it at their site and to do the same
  tests as AFNIC does (administrators of zones under
  ``.fr'' are encouraged to run ZC before submitting their
  request for creation/modification),
\item<5->Hosted at the hosting service Savannah,
\item<6->Completely IPv4 and IPv6,
\item<7->Used in daily production at AFNIC since.
\end{itemize}
\end{frame}

\begin{frame}
\frametitle{Engine, not policy}  
\begin{block}{Zonecheck is an engine, not a policy}{This is
    probably the main feature of Zonecheck: unlike all the other
    similar tools, the policy is not hardwired in the
    code.}\end{block}

\only<2->{The code defines all the tests you \emph{can} run, the
  configuration file defines the subset of the tests that you \emph{do} run and their
  result (fatal error or just a warning).}

\end{frame}

\begin{frame}[fragile]
\frametitle{Example of configuration}  
\begin{info}
      <check name="icmp" severity="w" category="connectivity:l3"/>
      <check name="udp"  severity="f" category="connectivity:l4"/>
      <check name="tcp"  severity="f" category="connectivity:l4"/>
\end{info}
\only<2->{A program can translate this configuration file to HTML, for
the information of the users.}
\end{frame}

\begin{frame}
  \frametitle{Using it to check delegations from a registry}
AFNIC uses Zonecheck \emph{prior} to every delegation. One fatal error
and the domain is not created. (Every name server change triggers a
Zonecheck, too.)

\only<2-3>{The policy is quite strict. A few examples:
  \begin{itemize}
    \item TCP connectivity is mandatory,
    \item If the server is recursive, a lot of tests occur (such as
    whether the loopback address is delegated in in-addr.arpa).
  \end{itemize}
}

\only<3->{As a side effect, this creates a
  large number of support tickets (that may be used to measure the
  current skills level of some registrars :-) and (without smiley) the
  current level of competence of many DNS administrators}

\only<4->{But it makes a much better zone and strongly diminishes the
  post-registration complaints of the type ``My site does not work''.}
\end{frame}

\begin{frame}
\frametitle{Lessons for IANA checks}
Context: IANA asks for comments about delegation checks
(\url{http://www.icann.org/announcements/announcement-18aug06.htm}).

[Generally speaking, the quality of DNS delegation is a very common
issue today.]

\only<2->{Many registries (CENTR, ccNSO) asked that such tests must be clearly
  described, and executed in a predictable way. An automatic tool,
  such as Zonecheck, fulfills these requirements.}

\only<3->{Remember that using Zonecheck does not mean using AFNIC policy.}
  
\end{frame}

\begin{frame}
\frametitle{Future tests?}
\begin{itemize}
\item DNSsec tests (see Eric Osterweil's slides)
\item ``OR'' tests: ``at least M among N nameservers'', ``TCP or
  EDNS0'', \ldots
\end{itemize}
\end{frame}

\begin{frame}
\frametitle{Other users}
\begin{itemize}
  \item \url{.de}
  \item \url{.ch}
  \item \url{.no}
\end{itemize}

\begin{block}<2->{Tomorrow, you?}{\url{http://www.zonecheck.fr/}}\end{block}

\end{frame}

\end{document}
